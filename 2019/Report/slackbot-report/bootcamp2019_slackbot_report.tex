\documentclass[12pt]{jsarticle}
\usepackage[dvipdfmx]{graphicx}
\textheight = 25truecm
\textwidth = 18truecm
\topmargin = -1.5truecm
\oddsidemargin = -1truecm
\evensidemargin = -1truecm
\marginparwidth = -1truecm

\def\theenumii{\Alph{enumii}}
\def\theenumiii{\alph{enumiii}}
\def\labelenumi{(\theenumi)}
\def\labelenumiii{(\theenumiii)}
\def\theenumiv{\roman{enumiv}}
\def\labelenumiv{(\theenumiv)}
\usepackage{comment}
\usepackage{url}

%%%%%%%%%%%%%%%%%%%%%%%%%%%%%%%%%%%%%%%%%%%%%%%%%%%%%%%%%%%%%%%%
%% sty/ にある研究室独自のスタイルファイル
\usepackage{jtygm}  % フォントに関する余計な警告を消す
\usepackage{nutils} % insertfigure, figref, tabref マクロ



\def\figdir{./figs} % 図のディレクトリ
\def\figext{pdf}    % 図のファイルの拡張子

\begin{document}
%%%%%%%%%%%%%%%%%%%%%%%%%%%%
%% 表題
%%%%%%%%%%%%%%%%%%%%%%%%%%%%
\begin{center}
{\LARGE SlackBotプログラムの報告書}
\end{center}

\begin{flushright}
  2019/4/26\\
  松尾 和樹
\end{flushright}
%%%%%%%%%%%%%%%%%%%%%%%%%%%%
%% 概要
%%%%%%%%%%%%%%%%%%%%%%%%%%%%
\section{はじめに}
\label{sec:introduction}
本資料は,2019 年度 B4 新人研修課題のSlackBotプログラム作成の報告書である.
新人研修課題としてSlackBotプログラムを作成した.
Slack\cite{slack}とは,チャットツールの一種である.
SlackBotとは,Slackのチャンネルに投稿を行ったり,
設定された契機となる単語によってユーザへの返信を自動で行ったりするものである.
本資料では,課題内容,理解できなかった部分,作成できなかった機能,および自主的に作成した機能について述べる.

\section{課題内容}
課題は以下の2つを行う.
\begin{enumerate}
\item 任意の文字列を投稿するプログラムの作成\\
ユーザから``(任意の文字列)と言って''という文字列を受信した際に,``(任意の文字列)''を返信するプログラムを作成する.
\item SlackBotプログラムへの機能追加\\
SlackBotプログラムへ機能を追加する.Slack以外のWebサービスのAPIやWebhookを利用した機能を追加する.
たとえば,検討打合せの3日前ならば予定を投稿する機能である.
本プログラムのコード量は,335行であった.
\end{enumerate}

\section{理解できなかった部分}
理解できなかった部分を以下に示す.
\begin{enumerate}
\item nokogiriを使用した際に発生したエラー\\
nokogiriを使用した際に,gemの依存関係によるエラーが発生した.
nokogiriのバージョンを1.10.2から1.6にするとエラーは発生しなかった.
nokogiriの適切なバージョンはわからなかった.
\end{enumerate}

\section{作成できなかった機能}
作成できなかった機能を以下に示す.
\begin{enumerate}
\item SlackのOutgoing Webhook以外からのPOSTリクエストをブロックする機能
\item しりとりにおいて,ユーザが投稿した単語の品詞を判定する機能
\item 一度使用した単語を再度使用できないようにする機能
\end{enumerate}

\section{自主的に作成した機能}
自主的に作成した機能を以下に示す.
\begin{enumerate}
\item コマンドにより,しりとりを開始または終了させる機能
\item コマンドにより,しりとり中に現在の状態を表示する機能\\
しりとりの状態とは,しりとりの途中か否かの情報とSlackBotが直前に投稿した単語の最後の文字である.
\end{enumerate}


\bibliographystyle{ipsjunsrt}
\bibliography{mybibdata}


\end{document}
